\documentclass[12pt]{beamer}

\usetheme{Darmstadt}

\usepackage[utf8]{inputenc}
\usepackage[T1]{fontenc}
\usepackage[english]{babel}

\usepackage{graphicx}    % include images
\usepackage{booktabs}    % nicer tables
\usepackage{amsmath,amssymb,amsthm} % math
\usepackage{listings}    % code
\usepackage{hyperref}    % links (beamer loads it with options)
\usepackage{microtype}   % better typography
\title{Logistic maps and chaos}
\author[Anton \and Vasyl]{
  Anton Valihurskyi\inst{1} \\
  Maksym-Vasyl Tarnavskyi\inst{1}
}
\institute[Inst.]{\inst{1} Faculty of Applied Sciences, UCU\\}
\date{}

\begin{document}

\begin{frame}[plain]
  \titlepage
\end{frame}

\begin{frame}{Outline}
  \tableofcontents
\end{frame}

\section{Introduction}
\begin{frame}{Problem statemet}
  Suppose we want to know how much rabbits we will have next year. \pause

  A fair assumption that next year's population depends on current.

  \begin{equation}
    x_{n+1} = rx_n
  \end{equation}
  Where $x_n$ is number (in, for example, thousands) of rabbits in year $n$ \pause
  But this way it will grow infinitely, so let us add limiting factor
  \begin{equation}
    x_{n+1} = rx_n(1-x_n)
  \end{equation}

\end{frame}

\begin{frame}{Logistic map}
  This discrete equivalent of logistic equation is called \textit{logistic map}.

  We restrict ourselves to values $r \in [0,4]$ and $x \in [0,1]$. This way logistic map maps to itself, which is natural in population growth model.

  \centering
  \includegraphics[width=0.8\textwidth]{../assets/logistic_curve.png}

\end{frame}

\begin{frame}{Period doubling}
  One might notice, that periods doubles as $r$ growth.

  \begin{table}[h!]
    \centering
    \begin{tabular}{c c}
      \hline
      $r_1 = 3$              & (period 2 is born) \\[4pt]
      $r_2 = 3.449\ldots$    & $4$ \\[4pt]
      $r_3 = 3.54409\ldots$  & $8$ \\[4pt]
      $r_4 = 3.5644\ldots$   & $16$ \\[4pt]
      $r_5 = 3.568759\ldots$ & $32$ \\[4pt]
      $\vdots$               & $\vdots$ \\[4pt]
      $r_\infty = 3.569946\ldots$ & $\infty$ \\
      \hline
    \end{tabular}
  \end{table}

  Interestingly, distance between consecutive transitions shrinks by constant factor
  \begin{equation}
    \delta = \lim_{n \to \infty} \dfrac{r_{n} - r_{n-1}}{r_{n+1} - r_{n}} \approx 4.669
  \end{equation}
\end{frame}

\section{Logistic map analysis}
\begin{frame}{Stability analysis}

  We call a fixed point $x^* = f(x^*)$, where $f$ generates map $x_{n+1} = x_n$ stable, if all points in its $\delta$-neighborhood converge to $x^*$.

  \begin{align}
    x^* + \eta_{n+1} &= x_{n+1} = f(x^* + x_n) \\&= f(x^*) + f'(x^*)\eta_n + o(\eta^2_n) \\
    \Rightarrow \eta_{n+1} = f'(x^*)\eta_n + o(\eta^2_n)
  \end{align}

  If $o(\eta^2_n)$ is neglectable, than $x^*$ is linearly stable if $|f'(x^*)| < 1$. If $|f'(x^*)| > 1$ than the $x^*$ is unstable. Otherwise stability is defined by $o(\eta^2_n)$.

\end{frame}

\begin{frame}{Growth rate and chaos}
  It is easy to see that fixed points are
  \begin{equation}
    x^*_1 = 0 \quad \text{and} \quad x^*_2 = 1 - \frac{1}{r}
  \end{equation}
  Where $x^*_2$ requires $r \ge 1$.

  Stability depends on $f'(x^*) = r -2rx^*$. It is easy to see that $x^*_1$ is stable for $r<1$ and $x^*_2$ is stable for $r \in (1,3)$,and unstable $r>3$.

  This hints us on why for $r>3$ the map does not converge.

\end{frame}

\begin{frame}{Growth rate and chaos}
  \centering
  \includegraphics[width=0.9\textheight]{../assets/r=1plot.png}
\end{frame}

\begin{frame}{Growth rate and chaos}
  As $r>1$ the slope at $x^*$ gets very steep. We can show that critical slope is $f'(x^*) = -1$ is attanined at $r=3$.

  Two periods exist $\iff$ $\exists p,q : f(p) = q \wedge f(q) = p$\\
  $\iff f(f(p)) = p$
  If we solve $f(f(x)) = x$ (polynomial of 4'th degree) and find roots besides $x^*_1, x^*_2$ we find that
  \begin{equation}
    p,q = \frac{r+1\pm\sqrt{(r-3)(r+1)}}{2r}
  \end{equation}
  For $r>3$ there are two distinct roots.Further more this 2-cycle deviation from $x^*$ is continuous in relation to $r$.
\end{frame}

\begin{frame}{Stability of 2-cycle bifurcation}
  We can check stability of 2-cycle by looking at $f(f(x))$ the same way as we did for $f(x)$.

  \begin{align}
    \lambda = \frac{\mathrm{d}}{\mathrm{d}x}(f(f(x)))_{x=p} = 4+2r-r^2 \\
    \Rightarrow |\lambda| < 1 \iff 3 < r < 1 + \sqrt{6} \approx 3.449
  \end{align}
\end{frame}

\begin{frame}{Periodic windows}
  From the orbit diagram, we could notice that after chaotic behavior, there is a window of period-3 ($3.8284.. < r < 3.8415$ ).
  Black dots correspond to stable period-3 cycle.

  \centering
  \includegraphics[width=0.8\textheight]{../assets/qubic.png}
\end{frame}

\begin{frame}
  Interestingly, we can find critical value of $r$ where period-3 window starts.

  To do it we need to find $r \in (3.8, 3.83)$ such that $y=x$ becomes tangent to $f(f(f(x)))$.

  Turns out solution can be found analytically, by considering following well-defined system:

  \begin{align}
    \begin{cases}
      y = r x (1 - x) \\
      z = r y (1 - y) \\
      x = r z (1 - z) \\
      \frac{\mathrm{d}(f^3(x))}{\mathrm{d}x}=r^3(1 - 2x)(1 - 2y)(1 - 2z) = 1
    \end{cases}
  \end{align}

\end{frame}

\begin{frame}

  Through simple subsitution and algebraic manipulation presented in "The birth of period three" (Partha Saha, Steven H. Strogatz 1994), it can be found that $r = 1 + \sqrt{8} = 3.8284...$.

  Which resembles $1+\sqrt{6}$.
\end{frame}

\begin{frame}{Ghost of period-3}

\end{frame}

\begin{frame}{Period doubling}
  Turns out, this value cannot be expressed using other mathematical constant, thus it gained its own name \textit{Feigenbaum constant}

  Feigenbaum was not good with computers, so he tried to calculate it manually with calculator.

\end{frame}

\end{document}