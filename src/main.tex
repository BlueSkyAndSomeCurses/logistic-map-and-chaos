\documentclass[12pt]{beamer}

\usetheme{Darmstadt}

\usepackage[utf8]{inputenc}
\usepackage[T1]{fontenc}
\usepackage[english]{babel}

\usepackage{graphicx}    % include images
\usepackage{booktabs}    % nicer tables
\usepackage{amsmath,amssymb,amsthm} % math
\usepackage{listings}    % code
\usepackage{hyperref}    % links (beamer loads it with options)
\usepackage{microtype}   % better typography
\usepackage{lmodern}
\title{Logistic maps and chaos}
\author[Anton \and Vasyl]{
  Anton Valihurskyi\inst{1} \\
  Maksym-Vasyl Tarnavskyi\inst{1}
}
\institute[Inst.]{\inst{1} Faculty of Applied Sciences, UCU\\}
\date{}

\begin{document}

\begin{frame}[plain]
  \titlepage
\end{frame}

\begin{frame}{Outline}
  \tableofcontents
\end{frame}

\section{Introduction}
\begin{frame}{Problem statemet}
  Suppose we want to know how much rabbits we will have next year. \pause

  A fair assumption is that the next year's population depends on the current.

  \begin{equation}
    x_{n+1} = rx_n
  \end{equation}
  Where $x_n$ is number (in, for example, thousands) of rabbits in the year $n$ \pause\\
  But this way it will grow infinitely, so let us add a limiting factor
  \begin{equation}
    x_{n+1} = rx_n(1-x_n)
  \end{equation}

\end{frame}

\begin{frame}{Logistic map}
  This discrete equivalent of logistic equation is called \textit{logistic map}.

  We restrict ourselves to values $r \in [0,4]$ and $x \in [0,1]$. This way logistic map maps to itself, which is natural in population growth model.

  \centering
  \includegraphics[width=0.8\textwidth]{../assets/logistic_curve.png}

\end{frame}

\begin{frame}{}
  \centering
  \vfill
  {\Huge Video}
  \vfill
\end{frame}

\begin{frame}{Period doubling}
  One might notice, that periods double as $r$ grows.

  \begin{table}[h!]
    \centering
    \begin{tabular}{c c}
      \hline
      $r_1 = 3$              & (period 2 is born) \\[4pt]
      $r_2 = 3.449\ldots$    & $4$ \\[4pt]
      $r_3 = 3.54409\ldots$  & $8$ \\[4pt]
      $r_4 = 3.5644\ldots$   & $16$ \\[4pt]
      $r_5 = 3.568759\ldots$ & $32$ \\[4pt]
      $\vdots$               & $\vdots$ \\[4pt]
      $r_\infty = 3.569946\ldots$ & $\infty$ \\
      \hline
    \end{tabular}
  \end{table}

  Interestingly, distance between consecutive transitions shrinks by constant factor
  \begin{equation}
    \delta = \lim_{n \to \infty} \dfrac{r_{n} - r_{n-1}}{r_{n+1} - r_{n}} \approx 4.669
  \end{equation}
\end{frame}

\begin{frame}{}
  \centering
  \vfill
  {\Huge Video}
  \vfill
\end{frame}

\section{Logistic map analysis}
\begin{frame}{Stability analysis}

  We call a fixed point $x^* = f(x^*)$, where $f$ generates map $x_{n+1} = x_n$ stable, if all points in its $\delta$-neighborhood converge to $x^*$.

  \begin{align}
    x^* + \eta_{n+1} &= x_{n+1} = f(x^* + \eta_n) \\&= f(x^*) + f'(x^*)\eta_n + o(\eta^2_n) \\
    \Rightarrow \eta_{n+1} = f'(x^*)\eta_n + o(\eta^2_n)
  \end{align}

  If $o(\eta^2_n)$ is neglectable, then $x^*$ is linearly stable if $|f'(x^*)| < 1$. If $|f'(x^*)| > 1$, then the $x^*$ is unstable. Otherwise stability is defined by $o(\eta^2_n)$.

\end{frame}

\begin{frame}{Growth rate and chaos}
  It is easy to see that fixed points are
  \begin{equation}
    x^*_1 = 0 \quad \text{and} \quad x^*_2 = 1 - \frac{1}{r}
  \end{equation}
  Where $x^*_2$ requires $r \ge 1$.

  Stability depends on $f'(x^*) = r -2rx^*$. It is easy to see that $x^*_1$ is stable for $r<1$ and $x^*_2$ is stable for $r \in (1,3)$, and unstable if $r>3$.

  This hints us on why for $r>3$ the map does not converge.

\end{frame}

\begin{frame}{Growth rate and chaos}
  \centering
  \includegraphics[width=0.9\textheight]{../assets/r=1plot.png}
\end{frame}

\begin{frame}{Growth rate and chaos}
  As $r>1$ the slope at $x^*$ gets very steep. We can show that critical slope $f'(x^*) = -1$ is attained at $r=3$.

  Two periods exist $\iff$ $\exists p,q : f(p) = q \wedge f(q) = p$\\
  $\iff f(f(p)) = p$.
  If we solve $f(f(x)) = x$ (polynomial of 4'th degree) and find roots besides $x^*_1, x^*_2$, we find that
  \begin{equation}
    p,q = \frac{r+1\pm\sqrt{(r-3)(r+1)}}{2r}
  \end{equation}
  For $r>3$ there are two distinct roots. Furthermore, this 2-cycle deviation from $x^*$ is continuous in relation to $r$.
\end{frame}

\begin{frame}
  \centering
  \vfill
  {\Huge Cobweb Video}
  \vfil

\end{frame}

\begin{frame}{Stability of 2-cycle bifurcation}
  We can check stability of 2-cycle by looking at $f(f(x))$ the same way as we did for $f(x)$.

  \begin{align}
    \lambda = \frac{\mathrm{d}}{\mathrm{d}x}(f(f(x)))_{x=p} = 4+2r-r^2 \\
    \Rightarrow |\lambda| < 1 \iff 3 < r < 1 + \sqrt{6} \approx 3.449
  \end{align}
\end{frame}

\begin{frame}{Periodic windows}
  From the orbit diagram, we could notice that after chaotic behavior, there is a window of period-3 ($3.8284.. < r < 3.8415$ ).
  Black dots correspond to stable period-3 cycle.

  \centering
  \includegraphics[width=0.8\textheight]{../assets/qubic.png}
\end{frame}

\begin{frame}
  Interestingly, we can find critical value of $r$ where period-3 window starts.

  To do it we need to find $r \in (3.8, 3.83)$ such that $y=x$ becomes tangent to $f(f(f(x)))$.

  Turns out solution can be found analytically, by considering following well-defined system:

  \begin{align}
    \begin{cases}
      y = r x (1 - x) \\
      z = r y (1 - y) \\
      x = r z (1 - z) \\
      \frac{\mathrm{d}(f^3(x))}{\mathrm{d}x}=r^3(1 - 2x)(1 - 2y)(1 - 2z) = 1
    \end{cases}
  \end{align}

\end{frame}

\begin{frame}

  Through simple subsitution and algebraic manipulation presented in "The birth of period three" (Partha Saha, Steven H. Strogatz 1994), it can be found that $r = 1 + \sqrt{8} = 3.8284...$.

  Which resembles $1+\sqrt{6}$.
\end{frame}

\begin{frame}{Ghost of period-3}
  $r = 3.8282$, just below period-3 window beginning

  \begin{center}
    \includegraphics[width=\textwidth]{../assets/ghost_of_period3.png}
  \end{center}

\end{frame}

\begin{frame}
  \centering
  \vfill
  {\Huge Cobweb Video}
  \vfil

\end{frame}

\section{Chaos}
\begin{frame}{Liapunov exponent for difference equation}
  For point $x_0$ consider point in its $\delta$-neighborhood $x_0 + \delta_0$.

  Let $\delta_n$ be distance between their $n$'th iterates.

  If $|\delta_n| \approx e^{\lambda n} |\delta_0|$, then $\lambda$ is called Liapunov exponent.

  Naturally, positive $\lambda$ leads to chaos. \pause

  Notice that $\delta_n = f^n(x_0 + \delta_0) - f^n(x_0) $ and
  \begin{align}
    \lambda &\approx \frac{1}{n}\ln{\left|\frac{f^n(x_0 + \delta_0) - f^n(x_0)}{\delta_0}\right|} \\
    &\overset{\delta_0 \rightarrow 0}{=} \frac{1}{n}\ln{\left| (f^n)'(x_0) \right|} \\
  \end{align}
  We can see that
  \begin{equation}
    \lambda \approx \frac{1}{n}\ln{\left| \prod_{i=0}^{n-1}f'(x_i) \right|} =  \frac{1}{n} \sum_{i=0}^{n-1}\ln{|f'(x_i)|}
  \end{equation}

\end{frame}

\begin{frame}{Liapunov exponent for logistic map}
  Formula (15) gives hints us on how to calculate Liapunov exponent to get a glimpse of chaotic behavior in logistic map.

  For each $r \in [3,4]$ calculate $\lambda$ for large number of iterations and initial point $x_0 = 0.5$.

  \centering
  \includegraphics[width=0.9\textheight]{../assets/lyapunov_exp.png}
\end{frame}

\begin{frame}{Period doubling}

  Turns out, logistic map is not the only map that exhibits period doubling route to chaos. This phenomenon is universal for a wide class of maps (so called unimodal maps).
  Basically, a smooth, concave down map with a single maximum.
  \centering
  \includegraphics[width=0.9\textheight]{../assets/unimodal.png}

\end{frame}

\begin{frame}{Period doubling}

  Moreover, for all unimodal maps $x_{n+1} = rf(x_n)$ with $f(0) = f(1) = 0$, the ratio of distances between bifurcation points approaches the same constant $\delta = \lim_{n \to \infty} \frac{r_{n} - r_{n-1}}{r_{n+1} - r_{n}} \approx 4.6692016...$.

  $$\Delta_n = r_{n+1} - r_{n}$$
  $$\frac{\Delta_n}{\Delta_{n+1}} \to \delta \quad \text{as } n \to \infty$$

  There is also universal scaling in the $x$-direction, but it is easier to show on the graph.

    \begin{equation*}
  \alpha = \lim_{n \to \infty} \frac{d_{n}}{d_{n+1}} \approx -2.5029078...
  \end{equation*}


\end{frame}

\begin{frame}{Period doubling}
  \centering
  \includegraphics[width=0.9\textheight]{../assets/constants.png}
\end{frame}

\begin{frame}{Self-Similarity (The Intuition)}
\begin{itemize}
    \item \textbf{The Problem:} Analyzing bifurcation points ($r_n$) is mathematically messy.
    \item \textbf{The Solution:} Analyze \textbf{Superstable Cycles} ($R_n$) instead.
    \item A cycle of period $2^n$ is \textit{superstable} when its multiplier is zero: $|(f^{2^n})'(x^*)| = 0$.
    \item This happens when the cycle passes exactly through the maximum of the function ($x_m$).
\end{itemize}
\end{frame}

\begin{frame}{Self-Similarity (The Intuition)}
  \textbf{The Comparison:} Compare the map $f(x)$ (fixed point at $R_0$) vs. the second iterate $f^2(x)$ (two fixed points at $R_1$).\\
  \centering
  \includegraphics[width=0.9\textheight]{../assets/self_similarity.PNG}
\end{frame}

\begin{frame}{Self-Similarity (The Intuition)}
  \textbf{Looking at $f^2$:} A $2$-cycle $p \to q \to p$ becomes two stable fixed points of $f^2(x)$.\\
  \textbf{"Zoom":} If we focus on one of those fixed points in $f^2$ (after appropriate parameter shift) and apply magnification/flip:
  \begin{center}
      $f^2(x, R_1)$ looks \textbf{identical} to $f(x, R_0)$.
  \end{center}
  \textbf{Conclusion:} The dynamics are the same, only scaled down.

\end{frame}

\begin{frame}{The Renormalization Operator}
  \centering
  \includegraphics[width=0.9\textheight]{../assets/renorm.png}

\end{frame}

\begin{frame}{The Renormalization Operator}
  \textbf{The Operator:} The Renormalization Operator $T$ formalizes the self-similarity by combining iteration and scaling:
  \begin{equation*}
  T[f(x, r)] = \alpha f(f(x/\alpha, r), r)
  \end{equation*}
  
  \textbf{Renormalization Step (3 operations):}
  \begin{enumerate}
      \item \textbf{Iterate:} Look at $f^2$ (dynamics of period $2^n$).
      \item \textbf{Shift:} Adjust $r$ to the next superstable value $R_{n+1}$.
      \item \textbf{Rescale:} Magnify the axes by a factor $\alpha$.
  \end{enumerate}
  
  \textbf{The Approximate Relation:}
  \begin{equation*}
  f(x, R_0) \approx \alpha f(f(x/\alpha, R_1), R_1)
  \end{equation*}
\end{frame}

\begin{frame}{The Renormalization Operator}
  \textbf{The Scaling Factor $\alpha$ (Alpha):}
  \begin{itemize}
      \item To match the graphs, the magnitude of scaling is:
      $$\alpha \approx -2.5029078...$$
      \item The negative sign indicates the branches are \textbf{flipped (inverted)} around the maximum $x_m$ at each doubling.
  \end{itemize}
  
  \centering
  $$\mathbf{f^2(x)} \xrightarrow{\text{Rescale by } \alpha} \approx \mathbf{f(x)}$$

\end{frame}

\begin{frame}{The Limit Function}

  \begin{itemize}
      \item If we repeat the Renormalization process $n \to \infty$ times, the functions converge to a \textbf{universal limit function} $g(x)$.
      \item This function satisfies the Functional Equation:
      \begin{equation}
      g(x) = \alpha g(g(x/\alpha))
      \end{equation}
  \end{itemize}

      \textbf{Universality Explained:}
      \begin{enumerate}
          \item The solution to the above depends \textbf{only} on the behavior of the function near its maximum (quadratic shape).
          \item It does \textbf{not} depend on the global shape of the original map ($f$).
      \end{enumerate}
\end{frame}

\begin{frame}{The Limit Function}
  The constants $\alpha$ and $\delta$ are derived from the properties of this universal function $g(x)$ and are therefore the same for all unimodal maps undergoing this period-doubling route to chaos.
  \vspace{1em}
  \centering
  \begin{itemize}
      \item $\boldsymbol{\delta \approx 4.669}$: Parameter scaling (eigenvalue).
      \item $\boldsymbol{\alpha \approx -2.503}$: Variable scaling.
  \end{itemize}
\end{frame}

\begin{frame}{1D Maps in Science}
  \begin{itemize}
      \item Real physical systems (e.g., convecting fluids, electronic circuits) have a huge number of degrees of freedom.
      \item However, if the system is \textbf{highly dissipative} (strong damping), only a few variables (e.g., 2 or 3) are truly \textbf{active}.
      \item The dynamics of these active variables can often be "captured" by a simple one-dimensional map.
  \end{itemize}

\end{frame}

\begin{frame}{1D Maps in Science}
  \centering
  \includegraphics[width=0.9\textheight]{../assets/rossler_attr.png}

\end{frame}

\begin{frame}
  \centering
  \vfill
  {\Huge Thank you for your attention!}
  \vspace{2em}
  \vfill
  {Sources: \\
      Strogatz, Steven H. \textit{Nonlinear Dynamics and Chaos: With Applications to Physics, Biology, Chemistry, and Engineering}. 2nd ed., Westview Press, 2015. \\
  \vfill
  }
\end{frame}

\end{document}